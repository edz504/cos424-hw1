\documentclass{article} % For LaTeX2e
\usepackage{cos424,times}
\usepackage{hyperref}
\usepackage{url}
\usepackage{graphicx}
\usepackage{amsmath}
\usepackage{multirow}
\usepackage{listings}
\usepackage{float}
\lstset{frame=tb,
  language=R,
  aboveskip=3mm,
  belowskip=3mm,
  showstringspaces=false,
  columns=flexible,
  basicstyle={\small\ttfamily},
  numbers=none,
  numberstyle=\tiny\color{gray},
  keywordstyle=\color{blue},
  commentstyle=\color{dkgreen},
  stringstyle=\color{mauve},
  breaklines=true,
  breakatwhitespace=true
  tabsize=3
}
\bibliographystyle{plos2009}


\title{COS424 Assignment 1: Email Classification}


\author{
Eddie D Zhou\\
\texttt{edzhou@princeton.edu} \\
\And
Daway Chou-Ren\\
\texttt{dchouren@princeton.edu} \\
}

\newcommand{\fix}{\marginpar{FIX}}
\newcommand{\new}{\marginpar{NEW}}

\begin{document}

\maketitle

\begin{abstract}
Spam has plagued the inboxes of email users for decades now, but it is still a fruitful machine learning exercise to attempt spam classification.  In this assignment, we address the problem of classifying emails into spam or ham using two different feature sets, and a variety of different classifiers, with the TREC 2007 spam dataset.  We find that logistic regression with bag-of-words features achieves the highest accuracy, but when examining different classifiers in a more refined and smaller feature space, a linear SVM achieves the best performance.
\end{abstract}
\section{Introduction}
We first use the given script to obtain bag-of-word features, and using cross-validation, we choose among Naive Bayes, Logistic Regression, and AdaBoost.  We re-train the best model on the full training set, and obtain the final test accuracy upon the testing set.  Next, we use custom features, extracted from the same dataset, and train the following models -- Random Forests, SVM (Gaussian), SVM (Sigmoid), SVM (Linear), Deep Neural Network, and Logistic Regression.  Picking the best model on validation score, we re-train said model on the full training set and obtain a testing accuracy again.\cite{guzella2009review, hastie2009elements}.  All Python and R code is available at our GitHub repo \cite{gh}.
\subsection{Data processing}
The TREC 2007 dataset was downloaded from the COS424 Piazza on 2/13/2015.  \cite{dataset} The test set had 5000 emails classified 50/50 as Spam/Haam, and the size 45000 training set was split 50/50.
\subsection*{Email Format}
Each individual email was stored as a text file, along with email metadata, with the general format as follows: sender information (address and date/time sent), return path information, receiver information, and message IDs, header information ("From", "To", etc.), and then the actual email body, encoded in plain text with HTML tags.
\subsection*{Data Extraction}
\par
Using the  \lstinline{email_process.py}\lstinline{} script (modifying the dictionary threshold to 100), we extracted enough word counts for each training and testing email to obtain 15228 features per email.\par 
The custom features are as follows and stored in this order, where $W$ is the total word count and $C$ is the total character count of an email:\cite{abbasi2008writeprints} \par 
\textbf{1:} hapax legomena / $W$\\
\textbf{2-9 Ratio of punctuation characters / $C$: }  , . ! ? ; : ' "\\
\textbf{11-32 Ratio of special characters / $C$:} \textasciitilde @ \# \$ \% \textasciicircum \& * ( ) \{ \} [ ] \textless \textgreater / \textbackslash - \_ + =\\
\textbf{33-37 Types of character ratios:} Uppercase characters / $C$, Lowercase characters / $C$, Digit characters / $C$, Special characters / $C$, Punctuation characters / $C$\\
\textbf{38-39:} Total characters, total words\\
\textbf{40-43 Time sent: } Hour, minute, second, and day of week (0 = Sunday)\\
\textbf{44-63 Word length counts:} Number of words length 1 / $W$, length 2, ... length 19, length 20+

Note: neither stemming nor tokenization was used to preserve character counts and punctuation.
\subsection{Classification methods}

We use three different classifiers from the Sci-Kit-Learn Python libraries for the vanilla feature set: \cite{scikit-learn}
\begin{enumerate}
\item \emph{Naive Bayes} (NB): Using the default parameters of MultinomialNB()
\item \emph{Logistic regression with $\ell_2$ penalty} (LOG): built on liblinear library
\item \emph{AdaBoost} (AdB): using $50$ decision trees as weak learners
\end{enumerate}

For the custom feature set, we use six different classification methods with various R packages: \cite{randomForest, meyer2014package}
\begin{enumerate}
\item \emph{Random Forest} (RF): default params of randomForest
\item \emph{Support Vector Machine (Gaussian)} (SVMG): default params
\item \emph{Support Vector Machine (sigmoid)} (SVMS): default params
\item \emph{Support Vector Machine (linear)} (SVML): default params
\item \emph{Deep Neural Network} (DNN): 100 epochs, tanh activation, 3 layers of 50 nodes, 50\% drop-out for each layer, 20\% of inputs dropped
\end{enumerate}
\subsection{Evaluation}
In the vanilla feature space, we used 5-fold cross validation on the full training set to obtain a simple accuracy metric -- we sum the total number of errors each model incurred on each validation fold, and subtract it from 1.\cite{kohavi1995study} In other words, we have the metric $A_q$ for model $q$:
\begin{align*}
A_q &= 1 - \frac{\sum_{i=1}^5(FP_{fi} + FN_{fi})}{N}
\end{align*}
where $FP_{fi}$ is the number of false positives obtained from training on all folds but fold $i$, and predicting on fold $i$ (likewise for the false negatives $FN_{fi}$.\par
For our experimental custom feature set, we simply set aside 20\% of the training set as a validation set.  We also obtained some more in-depth metrics rather than just the validation accuracy: namely, precision, recall, $F_1$-score, and log loss metrics, defined as
\begin{align*}
\text{P} &= \frac{TP}{TP + FP}, \hspace{1em} \text{R} = \frac{TP}{TP + FN}, \hspace{1em} F_1 = \frac{2\text{PR}}{\text{P} + \text{R}}, \hspace{1em} \text{log-loss} = -\frac{1}{N}\sum_{i=1}^N\sum_{j=1}^My_{i,j}\log(p_{i,j})
\end{align*}
where for the log-loss, there are $N$ samples, $M$ classes (2 in our case), $y_{i,j}$ is 1 if sample $i$ is in class $j$, and 0 otherwise, and $p_{i,j}$ is the model's probabilistic estimate of sample $i$ belonging in class $j$. \cite{powers2011evaluation}
\section{Results}
For our vanilla features and three models, we include the number of misclassified samples in each validation fold (indexing from f0 to f4 instead of f1 to f5), the average and total number of misclassifications, and the accuracy.
\begin{table}[h]
\centering
\small
\begin{tabular}{|c|c|c|c|c|c|c|c|c|}
\hline
    & f0 & f1 & f2 & f3 & f4 & average & total & gen\_acc \\ \hline
NB  & 29 & 18 & 21 & 38 & 25 & 26.2    & 131   & 0.997089 \\ \hline
LOG & 12 & 9  & 9  & 15 & 10 & 11.0    & 55    & 0.998778 \\ \hline
AdB & 13 & 19 & 15 & 26 & 21 & 18.8    & 94    & 0.997911 \\ \hline
\end{tabular}
\end{table}
While it's clear that the total validation accuracy is extremely high for all three methods, the Logistic Regression model obtains the best performance.  Taking the logistic regression as our best model, we re-train it on the full training set and then evaluate it in an unbiased fashion on the test set.  Doing so gives us an accuracy that is comparable to the generalization accuracy obtained during the validation phase -- \textbf{99.762\% test accuracy}.  Given this extraordinarily high test accuracy, we see a ROC curve that spikes up immediately, with a corner very near our perfectly desired $(0, 1)$ point. \cite{ggplot2}  Due to length restrictions, this ROC curve is included in our Github repo \cite{gh}.\par 
For the more refined feature space, defined by only 63 features, the is that with much lower space, time, and computational cost, we can achieve comparable performance.
\begin{table}[H]
\small
\centering
\begin{tabular}{|l|l|l|l|l|l|}
\hline
             & validation accuracy & log loss   & precision & recall    & $F_1$     \\ \hline
RF           & 0.5796667           & 0.3951222  & 0.8917411 & 0.1781494 & 0.2969708 \\ \hline
SVM\_gauss   & 0.4983333           & 0.1532875  & 0.4983333 & 1.0000000 & 0.6651835 \\ \hline
SVM\_sigmoid & 0.3371111           & 3.6592386  & 0.3623350 & 0.4345596 & 0.3951744 \\ \hline
SVM\_linear  & 0.8593333           & 0.6662509  & 0.8217070 & 0.9166109 & 0.8665683 \\ \hline
DNN          & 0.5195556           & 0.8806850  & 0.5176342 & 0.5268673 & 0.4307182 \\ \hline
Logistic     & 0.4610000           & 10.2988752 & 0.4563246 & 0.4263099 & 0.4493372 \\ \hline
\end{tabular}
\end{table}
It is clear that the linear support vector machine performed the best, with a much higher validation accuracy, precision, recall, and $F_1$ score.
\begin{figure}[ht]
\centering
\includegraphics[width=130mm]{roc_custom.png}
\caption{ROC curves for each classifier on the custom features validation set}
\end{figure}
The ROC curves confirm the numbers, regarding the linear SVM performance.  Re-training on the full training set, and evaluating on test gives \textbf{88.98\% test accuracy}, which is very comparable to the vanilla feature set given the lower costs.
%\subsection{Computational speed}
%As expected, all of the classifiers' had much shorter training times on the custom features set than the times on the higher-dimensional vanilla feature set.  Of the first three, Naive Bayes was quicker than the Logistic Regression, which was much quicker than AdaBoost (which requires training all the weak learner decision trees).  For the custom feature set, the support vector machines all took longer than the the other three models.
\section{Discussion}
\subsection{Vanilla}
Given the extremely high accuracy of the logistic regression on the vanilla feature datasets, it would be interesting to examine the incorrectly classified examples from that testing dataset.  We split the testing set into two parts -- the first being the 12 samples that were incorrectly classified, and the second being the 4988 samples that were correctly classified.  Taking the average over each features, grouped by the set, we then examine the differences between each of these mean feature arrays.  The analysis gives 12705 features where the incorrect samples had a greater mean value, and 1446 features where the correct sample average was higher.  The average of these differences was only 0.021 for when in the incorrect samples' favor, but a much larger 1.559 in the other direction.  This leads us to believe that our logistic regression model mainly had trouble with examples where the feature values (word counts) were relatively large.\par 
We can further examine the classifier's specifics regarding the incorrectly and correctly classified examples by looking at the decision function values.  The absolute value of the classifier's decision function values on the incorrect samples was 3.915, but on the correct samples, it was 17.188.  It makes sense that the decision function scores are high for the correctly classified samples -- in other words, the logistic model is very certain of the outputted classification on these samples.  On the other hand, the magnitude of the decision function values on the incorrectly classified samples is much smaller, reflecting the uncertainty the model has in classifying them. \cite{zhu2009}\par
%Also, one might wonder why a linear SVM was not tested for the larger vanilla feature set -- the rationale was that given the extraordinarily high validation and test accuracies in the models trained on the vanilla feature set, there would not be a large difference in performance.  This leads us to believe that even if we were to spend the huge amount of time necessary to train the linear support vector machine in 15228-space, the gain in accuracy over the more parsimonious and efficient logistic regression would be negligible.
\subsection{Custom Features}
To analyze which custom features were most distinguishing, we took the difference of average feature values for the sets of emails classified as spam and not spam, divided by the average variance, and sorted in decreasing order. The most distinguishing features turned out to be mostly special character frequencies, such as '/', '\_', '"', '\$', '\#', '\^', ',', and '@'. Of the ten most distinguishing features, the first eight were special characters, and the last two were the total number of characters and the day of the week the email was sent. \par
The influence of special character frequencies makes sense since these are most likely very infrequently used in non-spam messages. Even a relatively low usage in spam messages, perhaps because they are automated may give them away. We also expect time features like the day of the week to prove useful since spammers might use automated tools to batch send emails at certain times.

\section{Conclusion and Extensions}
In summary, using a slightly modified version of the bag-of-words features given to us alongside a logistic regression allows us to obtain superb testing accuracy of 99.762\%.  With a much smaller, custom-made feature space, the validation phase gives us a linear support vector machine, which produces a testing accuracy of 88.98\% at a much smaller space cost.\par 
A natural extension would be to train the linear SVM on the larger vanilla-feature dataset, but as mentioned earlier, the potential gains in accuracy would be negligible at best given the already superb (and presumably quicker) performance of other models on that data.\par 
Another extension that we believe could be fruitful is hyperparameter tuning for the deep neural network -- since these methods are highly parameter-dependent, tuning of the number of epochs, layers, etc. could result in performance that outperforms the linear SVM.  This would be highly desirable, given the shorter training time for the DNN over the SVM.

From the testing accuracies of almost 90\% using our much smaller feature set, it seems possible that a particularly well-refined feature set could generate results as good as the bag-of-words. Some ideas we would explore if given more time would be to full use the metadata associated with each email to generate features. Given the high distinguishing power of special characters, we would like to do character and token analysis on the email addresses of senders as well as on the subject lines. Extending our set of syntactic and lexical features to include things like hapax dislegomena, types of capitalization used, letter n-grams, word n-grams, and even part of speech tags may also prove useful.  We are also interested by the possibility of analyzing the structure of emails themselves (the number of paragraphs, length of paragraphs, and even the type and amount of whitespace).\cite{matthews1993neural, merriam1994neural, holmes1995federalist} \par
Another interesting approach may be to break up the set of non-spam emails through clustering. Emails sent to co-workers and bosses will be different both stylistically and structurally than emails sent to friends. It is possible that spam emails closely resemble one type of email. It may be useful to identify clusters of emails that are more similar to spam than others and extract features that have high distinguishing power for these clusters.



\bibliographystyle{plain}
\bibliography{ref}


\end{document}